
\section{Introduction}

The basis of this lecture course has been the parallelisation of various numerical methods, with a particular focus on the Finite Element Method. For this project we have been supplied with a software package (which we shall henceforth call \emph{the FEM package}) that computes the solution to the Poisson partial differential equation using the Finite Element Method. This package is self contained - it includes its own custom matrix/vector types, and its own implementations of linear algebra operations on these types. Our goal is to make improvements to the package with the help of the FLENS.

FLENS (\emph{Flexible Library for Efficient Numerical Solutions}) is a C++ library written by Dr. Michael Lehn, which offers a comprehensive collection of matrix and vector types. Included is a C++ -based BLAS (\emph{Basic Linear Algebra Subfunctions} implementation, which provides linear algebra operations, such as matrix-vector multiplication, on these types. 

The advantages of using the FLENS in the FEM package are numerous. Firstly, use of an external library for matrix/vector types adds some standardisation to the package, aiding, for example, a user who is new to the FEM package, but has experience of the FLENS from other projects. Secondly, the library can be linked, with almost trivial effort, to any optimised BLAS or LAPACK libraries available, such as \emph{ATLAS} or \emph{Goto BLAS}, for instant speed improvements of BLAS operations. Thirdly, the FLENS offers overloaded operators for the BLAS operations. We recognise that users from different backgrounds may have a preference regarding the noation used for linear algebra operations, be it the tradition BLAS notation:
\begin{lstlisting}
   blas::mv(NoTrans, One, A, p, Zero, Ap);
\end{lstlisting}
or a notation more akin to that of MATLAB:
\begin{lstlisting}
   Ap = A*p;
\end{lstlisting}
(for matrix A, vector p). With FLENS, we have the choice.

We therefore summarise the aims of this project as follows:
\begin{enumerate}
   \item Replace all data storage objects with FLENS-based objects.
   \item Where possible, replace linear algebra operations with their BLAS equivalents, for speed-up from optimised BLAS libraries.
   \item Offer two versions of solvers, one with BLAS notation, one with overloaded operators.
\end{enumerate}

Thus it is worth noting that we are primarily improving the existing implementation from a software design point of view, with possible performance improvements, rather than adding new methods.

\section{Matrix and Vector Types}

A major part of converting the FEM package to the FLENS framework is the transition from the package's custom storage types to FLENS-based types. Of course, some types, such as the package's \emph{Vector} class, have exact FLENS equivalents. Others, however, contain bespoke objects and methods for the MPI communications. Thus we must create our own custom storage types in these cases.

\subsection{Equivalent Types and Index Base}

We adopt the following direct conversions from the FEM package to FLENS framework:
\begin{equation*}
\begin{array}{rcl}
   \mbox{\texttt{Vector}}  &  \rightarrow  &  \mbox{\texttt{DenseVector\<Array\<\double\> \>}} \\
   \mbox{\texttt{IndexVector}} & \rightarrow & \mbox{\texttt{DenseVector\<Array\<\int\> \>}}
\end{array}
\end{equation*}

However, we must note that the default index base in FLENS, which we are using here, is 1, as opposed to that of the package, which is 0. We make this change, despite the awkwardness it adds to the transition, because this is the natural index base regarding the mesh geometry. Numbering of the mesh nodes starts at 1, and assembly of the system of linear equations frequently accesses vector elements corresponding to node identities. For example in the FEM package:

\begin{lstlisting}
   someVec_FEMpackage(nodeID - 1) = someValue;
\end{lstlisting}

as opposed to using FLENS:

\begin{lstlisting}
   someVec_FLENS(nodeID) = someValue;
\end{lstlisting}

Whilst this is purely cosmetic, it may help to avoid bugs caused by forgetting to subtract 1 from node identities. For consistency, we implemented all FLENS matrices/vectors with index base 1.

FLENS includes a storage scheme for sparse matrices of CRS (Column-Row-Storage) type, offering an alternative to the FEM package's CRSMatrix type. However, these must be initialised from a FLENS sparse matrix with a coordinate storage scheme, effecting a change in the implementation of the type, but not requiring the creation of a custom type.

??? Mazen's CRS sorting?

\subsection{TypeI and TypeII}\label{subsc:typeI_II}

The FEM package uses the nomenclature defined in the lecture course, of \emph{TypeI} and \emph{TypeII} to distinguish between vectors that contain values corresponding to the problem posed on a compute node's local mesh, or on the global mesh:

\begin {itemize}
   \item \textbf{TypeI}: global values.
   \item \textbf{TypeII}: local values.
\end{itemize}

We adopt this definition in this paper and in our code, and refer to these types as \emph{MPI vector types}.

\subsection{FLENSDataVector} (Christopher Davis)

The FEM package's \emph{DataVector} class is the package's primary custom vector type. Included members are:
\begin{itemize}
   \item Vector object: stores the values of the DataVector.
   \item Coupling object: contains mesh geometry information required for MPI communications.
   \item A vectorType enumerated type: determines the MPI vector type of the vector (see Section \ref{subsc:typeI_II}).
\end{itemize}

We propose a FLENS-based replacement for this class called FLENSDataVector, which incorporates a few small yet profound modifications to the structure.

Firstly, we make the obvious choice of using a FLENS \texttt{DenseVector\<Array\<\double\> \>} to store our vector values. However, we choose to \emph{derive} our class from this FLENS type, rather than specifying a \texttt{DenseVector} as a member object. I.e. we use a `is-a' \texttt{DenseVector} approach, rather than a `has-a' approach. The advantage here is that our class inherits all methods and overloaded operator from the \texttt{DenseVector} class, and can be passed directly to the FLENS BLAS functions. This lends itself to a more parsimonious implementation - under the `has-a' approach we would need to overload every BLAS function, such as: 
\begin{lstlisting}
   double
   dot(FLENSDataVector x, FLENSDataVector y) {
   	blas::dot(x.vec, y.vec);
   }
\end{lstlisting}

and clutter our FLENSDataVector class with trivial operators, such as:

\begin{lstlisting}
   double &
   operator()(int index) {
   	return vec(index);
   }
\end{lstlisting}

neither of which are desirable.

The Coupling object is stored as a constant reference - the same way as in the FEM package.

Next we consider the enumerated type that specifies the MPI vector type. Here we wish to change the structure somewhat - whilst setting the MPI vector type in this fashion is easy and flexible, allowing the type to be changed as and when required, we find this to be \emph{too} flexible, and not ideal from a software design perspective. To help explain the situation, we consider a blind man and his socks. The man in question loves to wear socks, and therefore does so at all times. He commands an extensive collection, consisting of many different colours. Each morning he changes his socks, taking care to pair the dirty socks together so that they remain paired after washing. His system appears to work well - he always wears matching socks, and as a result leads a successful life. But what about if one morning whilst in the middle of changing his socks he is distracted by the telephone ringing, and as such forgets to change one of his socks - now his socks don't match! Whilst we would like to think that some kind person may notify him of his mistake, the world can be a cruel place. Odd socks are rarely tolerated by modern societies, so we can imaging that his life's achievements would probably crumble around him. Returning to the world of the FEM package, we hope that the difficulties that could arise from the enumerated type are clear - there is no way of determining whether the values contained in a TypeI DataVector are actually global values. There is always the chance that some rogue function changed the type without modifying the values. Such a problem would not make for an enjoyable debugging task.

Thus our FLENSDataVector is implemented as a template class, requiring the MPI vector type to be defined (permanently) at instantiation. The following classes are permitted as specialisation types:
\begin{itemize}
   \item[-] \texttt{class FLvNonMPI}
   \item[-] \texttt{class FLvTypeI}
   \item[-] \texttt{class FLvTypeII}
\end{itemize}

Clever implementation of the FLENSDataVector constructors limits specialisation of the class to these types \emph{only}, as well as ensuring the specification of a Coupling object for MPI types (and not for the non-MPI). We use specialisations of the constructor function for these types, for example:
\begin{lstlisting}
template <>
FLENSDataVector<FLvNonMPI>::FLENSDataVector(int n)
	: 	DenseVector<Array<double> >(n),
		coupling(Coupling())
{
	//Permits instatiation of FlNonMPI specialisation.
}
\end{lstlisting}

and we add a line to the unspecialised constructor that will cause an error at \emph{compile time} if scope ever reaches there (which would be due to a wrong type specialisation):
\begin{lstlisting}
template <typename VTYPE>
FLENSDataVector<VTYPE>::FLENSDataVector(int n)
	:	coupling(Coupling())
{
	VTYPE::CHK;			//<-- If scope ever reaches here,
					//compilation will fail.
			//e.g. if FLENSDataVector<double> instantiated.
}
\end{lstlisting} 

Thus the following instantiation would cause a compiler error:
\begin{lstlisting}
FLENSDataVector<double>  oops(5);
\end{lstlisting}

As such, we have limited the potential for undefined behaviour. 

All communication-related member methods of the FEM package's DataVector are added to the FLENSDataVector with no significant changes. For conversion methods, we require the object to be of the destination type. For example:
\begin{lstlisting}
FLENSDataVector<FLvTypeI> myVec(5, Coupling());

////////////////////////////////////
// *** fill with local values *** //
////////////////////////////////////

myVec.typeII_2_I();		<-- perfect
//myVec.typeI_2_II();		<-- would cause compiler error
\end{lstlisting}

\subsection{BLAS Overloading}

Most BLAS functions can be used in the FEM package in their usual form. The copy and dot product functions, however, require attention.

Copying between two FLENSDataVectors of the same MPI vector type can use the BLAS copy function without further assistance. The types match, and we can't do anything about the constant reference to the Coupling object (this must be left to the user to ensure). 

When copying vectors of differeing MPI vector types, we overload the BLAS copy function. Within this overloaded function, we call the BLAS copy function to copy the vector values by \emph{upcasting} the FLENSDataVectors to their parent class \texttt{DenseVector\<Array\<\double\> \>}, then perform the appropriate conversion, for example:
\begin{lstlisting}
void
copy(FLENSDataVector<FLvTypeII> &orig, FLENSDataVector<FLvTypeI> &dest) 
{

	//Copy data as usual (masquerading as a DenseVector :) ):
	blas::copy(*static_cast<DenseVector<Array<double> > *>(&orig),
		   *static_cast<DenseVector<Array<double> > *>(&dest));

	//Perform vector type conversion:
	dest.typeII_2_I();
}
\end{lstlisting}

We use a similar technique for the dot product - the dot product of the two supplied vectors is calculated using BLAS via upcasting, and then the appropriate MPI communication is performed.

By ensuring that our overloaded functions still use the FLENS BLAS implementation, we maintain the possibility for objective (2) in the Introduction.

We point out that our use of `proper' object type to define the MPI vector types means that the many type asserts present in the FEM package's linear algebra subroutines, such as:
\begin{lstlisting}
double dot(DataVector &u, DataVector &v)
{
	if(u.type==nonMPI && v.type==nonMPI) 
	  return u.values.dot(v.values);
	
	// we only multiply typeI and typeII vectors
	assert(u.type != v.type);		//<--assert
\end{lstlisting}
are not required. All type checking is moved to compile time, a significant advantage in terms of both runtime efficiency and ease of debugging.

\subsection{The CG Solver}

\subsubsection{Implementation}

In this section we look at the implementation of the conjugate gradient method for solving a system of linear equations. The solver was initially integrated into the FEM package via a wrapper, the details of which are described below in Section \ref{subsc:GS_solver}.

The CG solver uses many linear algebra operations, and is therefore a prime candidate for using BLAS functions. For example, we replace FEM package lines such as:
\begin{lstlisting}
CRSmatVec(p,A,x);
add(r2,p,-1.);
\end{lstlisting}
with the more universally recognised:
\begin{lstlisting}
blas::mv(NoTrans, One, A, x, Zero, p);
blas::axpy(-One, p, r2);
\end{lstlisting}

As claimed in objective (3) in the Introduction, we also offer a version of the CG solver where such BLAS functions are overloaded, providing a MATLAB-style notation:
\begin{lstlisting}


This class stores its vector values in 

Furthermore, now that we have a 