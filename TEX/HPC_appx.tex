%Appendix
\newpage
\appendix
\section{Appendix}

In this appendix we offer instructions on how to set up and run our software package on both a desktop computer and on the Pacioli compute cluster. We assume the target device is of the UNIX family, and already has the following:

\begin{itemize}
   \item gcc of at least version 4.7
   \item git
   \item MPI
\end{itemize}

\subsection{Desktop}

\begin{enumerate}
   \item Make a new directory where you would like the magic to happen.
   \shellcmd{mkdir FEMtastic}
   \item Add our package (the directory called HPC\_FLENS) to this new directory.
   \item Get and build FLENS: 
   \shellcmd{git clone https://github.com/michael-lehn/FLENS.git}
   \shellcmd{cd ./FLENS}
   \shellcmd{make}
   \shellcmd{cd ..}
   
   \item Our makefiles assume FLENS is located under /usr/local/FLENS, so copy it there. If this isn't possible, the makefiles will require minor modification.
   \shellcmd{cp -rvf ./FLENS /usr/local/FLENS}
   
   \item Build our package:
   \shellcmd{cd ./HPC\_FLENS}
   \shellcmd{make}
   
   \item Run our package.\\
   Replace n with the number of mesh refinements you desire (less than 8 is sensible), and cg can be replaced by gs if you have a lot of time on your hands:
   
   \begin{enumerate}
   
      \item Serial implementation:
      \shellcmd{./main-serial ./input/Square\_one\_domain n cg 1}
      \item Parallel implementation:
      \shellcmd{mpirun -np 4 main-parallel ./input/Square\_four\_domains n cg 1}
      
   \end{enumerate}
   
   \item Sit back and watch it work its magic.
\end{enumerate}
      
\subsection{Pacioli}

\begin{enumerate}
   \item Log in to Pacioli and make a new directory where you would like the magic to happen.
   \shellcmd{mkdir FEMtastic}
   \item Use scp to add our package (the directory called HPC\_FLENS) to this new directory.
   \item Get and build FLENS: 
   \shellcmd{git clone https://github.com/michael-lehn/FLENS.git}
   \shellcmd{cd ./FLENS}
   \shellcmd{make}
   \shellcmd{cd ..}\\
   (no copying of FLENS to /usr/local is required here, our makefiles know we're now on Pacioli $\ddot\smile$).
   
   \item Load modules:
   \shellcmd{module load gcc/4.7.2}
   \shellcmd{module load sge/6.2u5}
   \shellcmd{module load openmpi/gcc/64/1.4.2}
   
   \item Build our package:
   \shellcmd{cd ./HPC\_FLENS}
   \shellcmd{make}
   
   \item Submit the appropriate jobscript to the queue (modify the commands in it first, if you so wish):
   
   \begin{enumerate}
   \item Serial:
   \shellcmd{qsub batchscript\_hpc\_flens\_serial.qs}
   
   \item Parallel:
   \shellcmd{qsub batchscript\_hpc\_flens\_parallel.qs}
   
   \end{enumerate}
   
   \item Wait for the job to be processed. Since the parallel version demands exclusive use of 4 nodes, this may take some time if the queue is busy. Once complete, there will be an output file created in the HPC\_FLENS directory.
      
      
 \end{enumerate}
 
 \subsection{GotoBLAS}
 
 Notes:
 \begin{itemize}
    \item GotoBLAS offers highly machine-specific optimisations. We performed testing on the single processor (AMD Opteron 252) nodes of Pacioli (nodes 1-16). The frontend is identical to these nodes, and therefore can be used to build GotoBLAS. Creativity would be required to build the library for the other nodes.
    \item We found some issues with certain nodes (prime suspect: Node 16), whereby computation times were increased significantly ($\sim$10x) when using GotoBLAS. We suspect that this is due to a node that isn't actually identical to the others, and as such the `optimised' code is in fact the opposite. Therefore care should be taken to use the GotoBLAS library on processors for which it was configured/built.
 \end{itemize}
 To link FLENS with GotoBLAS (on Parcioli), perform steps 1-4 of A.2. Then:
 
 \begin{enumerate}
    \item Get GotoBLAS in the form of OpenBLAS:
    \shellcmd{git clone git://github.com/xianyi/OpenBLAS}
    \item Make OpenBLAS:
    \shellcmd{cd ./OpenBLAS}
    \shellcmd{make FC=gfortran}
    \shellcmd{cd ..}
    \item Install OpenBLAS:
    \shellcmd{mkdir gotoins}
    \shellcmd{cd ./OpenBLAS}
    \shellcmd{make PREFIX=../gotoins FC=gfortran install}
    \shellcmd{cd ..}
    \item Add OpenBLAS to the linker directory path:
    \shellcmd{export LD\_LIBRARY\_PATH=/home/\_\_your username\_\_/gotoins/lib:\$LD\_LIBRARY\_PATH}
    
    \item Build our package with GotoBLAS options:
    \shellcmd{cd ./HPC\_FLENS}
    \shellcmd{make goto}
    \item Submit the appropriate jobscript with `goto' in its name, in the same way as in A.2.
\end{enumerate}
    
    
    
    
    
    
    
    
    
    
    
      
      