
\documentclass[10pt, a4paper, mathserif]{beamer} %Standardklasse f�r Beamer 
% Option mathserif f�hrt zu mathematischen Zeichen mit Serifen

% Sprache und Kodierung
\usepackage[T1]{fontenc}
\usepackage[ngerman,english]{babel}
%\usepackage[ngerman]{babel} % Umlaute (wahlweise )
%\usepackage[latin1]{inputenc} % F�r Windows
\usepackage[applemac]{inputenc} % F�r Mac

%chris
%\usepackage{caption}
%\usepackage{animate}
\usepackage{tikz}
\usepackage{graphicx}
\usepackage{pgfplots}
\usetikzlibrary{plotmarks}
%\usepackage{epsfig} 
%\usepackage{epstopdf}

% code
%%%%%%%%%%%%%%%%%%%%%%%%%%%%%%%%%%%%% MAZEN %%%%%%%%%%%%%%%%%%%%%%%%%%%%%%%%%%%%%%%%%%%%%%%%
%\usepackage{algorithm}
%\usepackage{algpseudocode}
\usepackage[ruled,vlined]{algorithm2e}
\usepackage{listings}
\usepackage{xcolor}

\usetikzlibrary{shapes,arrows}
\usetikzlibrary{positioning}

\definecolor{myback}{HTML}{FFFFCC}
\lstset{language=C++,
		backgroundcolor=\color{myback},
		breaklines=true,
		commentstyle=\itshape\color{green!40!black},
		frame=L,
		numbers=left,
		basicstyle=\footnotesize\ttfamily,
		identifierstyle=\color{blue},
		stringstyle=\color{orange},
		keywordstyle=\color{red}}


%%%%%%%%%%%%%%%%%%%%%%%%%%%%%%%%%%%%% MAZEN %%%%%%%%%%%%%%%%%%%%%%%%%%%%%%%%%%%%%%%%%%%%%%%%



\newcommand{\<}{$\langle$}
\let\>=\undefined
\newcommand{\>}{$\rangle$}
\newcommand{\double}{{\color{blue}double}}
\let\int=\undefined
\newcommand{\int}{{\color{blue}int}}

%\usepackage{placeins}
\pgfplotsset{compat=1.3}
%\usepackage{subcaption}

% Schriftart 
\usepackage{avant} % serifenlos

% Mathematische Symbole und Umgebungen
%\usepackage{amsmath, amsthm, amsfonts}
\usepackage{mathtools}
% \usepackage{amsmath} % loaded by mathtools
\usepackage{amsthm}

% Grafiken
\usepackage{graphicx,ae}

% Erm�glicht das Einbinden von Filmen
\usepackage{multimedia}

%\usepackage{array}

% math fonts
\usepackage{amssymb}
\usepackage{dsfont}


% definition of math operators
\DeclareMathOperator{\dist}{dist}
\DeclareMathOperator*{\argmin}{argmin}
%\DeclareMathOperator*{\argmax}{arg\,max}
\DeclareMathOperator{\udN}{u. d. N.}
\DeclareMathOperator{\dom}{dom}
\DeclareMathOperator{\epi}{epi} % epigraph
\DeclareMathOperator{\interior}{int} % interior
\DeclareMathOperator{\closure}{cl} % closure
\DeclareMathOperator{\aff}{aff} % affine hull
\DeclareMathOperator{\ri}{ri} % relative interior
\newcommand{\argmax}[1]{\underset{#1}{\operatorname{arg}\,\operatorname{max}}\;}

% define paired delimiters
\DeclarePairedDelimiter\abs{\lvert}{\rvert}
\DeclarePairedDelimiter\norm{\lVert}{\rVert}
\DeclarePairedDelimiter\inner{\langle}{\rangle}
\DeclarePairedDelimiter\set{\lbrace}{\rbrace}
\DeclarePairedDelimiter\ceil{\lceil}{\rceil}
\DeclarePairedDelimiter\floor{\lfloor}{\rfloor}

% define symbolcs
\newcommand{\NN}{\ensuremath{\mathds{N}}}  % natural numbers
\newcommand{\ZZ}{\ensuremath{\mathds{Z}}}  % whole numbers
\newcommand{\QQ}{\ensuremath{\mathds{Q}}}  % rational numbers
\newcommand{\RR}{\ensuremath{\mathds{R}}}  % real numbers
\newcommand{\CC}{\ensuremath{\mathds{C}}}  % complex numbers
\newcommand{\cC}{\ensuremath{\mathcal{C}}} % continuous functions
\newcommand{\cN}{\ensuremath{\mathcal{N}}} % calligraphic N
\newcommand{\cB}{\ensuremath{\mathcal{B}}} % bundle
\newcommand{\cU}{\ensuremath{\mathcal{U}}} % u-space
\newcommand{\cV}{\ensuremath{\mathcal{V}}} % v-space
\newcommand{\cG}{\ensuremath{\mathcal{G}}} % graph
\newcommand{\cE}{\ensuremath{\mathcal{E}}} % edge set
\newcommand{\PD}{\ensuremath{S^n_{++}}}     % positive definite symmetric n-by-n matrix
\newcommand{\PSD}{\ensuremath{S^n_{+}}}     % positive semi-definite symmetric n-by-n matrix
\newcommand{\BB}{\ensuremath{\mathds{B}}}  % unit ball
\newcommand{\ones}{\ensuremath{\mathds{1}}}
\newcommand{\RRp}{\ensuremath{(\RR \cup \set{+\infty})}}
\newcommand{\RRn}{\ensuremath{(\RR \cup \set{-\infty})}}
\newcommand{\RRpn}{\ensuremath{(\RR \cup \set{-\infty,+\infty})}}
\newcommand{\Ff}{\ensuremath{\mathcal{F}}} %curly F
\newcommand{\FF}{\ensuremath{\mathds{F}}} %filtration F
\newcommand{\PP}{\ensuremath{\mathds{P}}} %double barrel P
\newcommand{\Tt}{\ensuremath{\mathcal{T}}} %curly T
\newcommand{\EE}{\ensuremath{\mathds{E}}} %expectation
\newcommand{\Rr}{\ensuremath{\mathcal{R}}} %regression
\newcommand{\Nn}{\ensuremath{\mathcal{N}}} %normal distribution

%%%%%%%%%%%%%%%%%%%%%%%%%%%%%%%%%%%%%%%%%%%%%%%%%%%%%%%%%%%%%%%%%%%%%%%%%%%%%%%%

\usetheme{Boadilla} % Pr�sentationsthema, bestimmt die Details einer Folie

\definecolor{ecs100}{RGB}{86,170,28}
\setbeamercolor{structure}{fg=ecs100,bg=white}
\setbeamercolor{alerted text}{fg=ecs100}
\setbeamertemplate{section in toc}[default]

\setbeamertemplate{itemize items}[circle]
\setbeamertemplate{enumerate items}[default]
\setbeamertemplate{navigation symbols}{}

\setbeamertemplate{headline}
{%
\vspace*{2.5ex}%
\begin{beamercolorbox}[wd=0.09\textwidth,ht=2ex,dp=0.5ex,leftskip=.5em,rightskip=.5em]{author in head/foot}%
\usebeamerfont{author in head/foot}%
Page \insertframenumber%
\end{beamercolorbox}%
\vspace*{-1.75ex}%
\hspace*{0.07\textwidth}%
\begin{beamercolorbox}[wd=0.92\textwidth,ht=2ex,dp=0.5ex,right,leftskip=.5em]{title}%
\begin{picture}(0,0)
\put(0,0.5){\rule[1.8ex]{\textwidth}{0.2ex}}\end{picture}% 2ex - 0.2ex = 1.8ex
{\usebeamerfont{title in head/foot}%
\qquad \insertshorttitle \hfill \insertsubsection $\leftarrow$\insertsection \break}
\end{beamercolorbox}%
}

\setbeamertemplate{frametitle}
{%
\vspace*{1ex}%
\usebeamercolor{title} \bf \large \insertframetitle%
\vspace*{-0.5ex} }

\setbeamertemplate{footline}{}

\setbeamertemplate{navigation symbols}{}

%%%%%%%%%%%%%%%%%%%%%%%%%%%%%%%%%%%%%%%%%%%%%%%%%%%%%%%%%%%%%%%%%%%%%%%%%%%%%%%%

\title{A Reimplementation of the Finite Element Method Software Package using FLENS}
\author{Mazen Ali, Christopher Davis}
\institute{University of Ulm}
\date{\today}

\setcounter{framenumber}{-1}

%MY DEFINITIONS BEGIN
\def\r#1{{\color{red}#1}}
\def\g#1{{\color{green}#1}}
\def\mg#1{{\color{ecs100}#1}}
\def\titlefont#1{\textbf{\large\mg{#1}}}

%\newtheorem{corollary}{Corollary}
\newtheorem{rem}{Remark}
\newtheorem{defi}{Definition}
%MY DEFINTIONS END

%%%%%%BEGIN DOCUMENT%%%%%%%%%%%%%%%%%%%%%%%%%%%%%%%
\begin{document}

%%%%%%%%%%%%%%%%%%%%%%%%%%%%%%%%%%%%%%%%%%%%%%%
%Titel Seite---------------------------------------------------------------------------------------------------------------------------
%%%%%%%%%%%%%%%%%%%%%%%%%%%%%%%%%%%%%%%%%%%%%%%
\frame[plain]{
  \begin{flushright}
  \includegraphics[width=0.4\linewidth]{Grafiken/UniLogo.pdf}
  \end{flushright}
  \vfill
  % Hier k�nnt ihr ein Titelbild einbinden, m�sst ihr aber nicht
  \vfill
  \parbox{0.24\linewidth}
  {
    \tiny \insertauthor\\ \insertinstitute\\ \insertdate
  }
  \parbox{0.75\linewidth}
  {
    \textcolor{ecs100}{\large\inserttitle}\\[0.2cm]
    \textcolor{ecs100}{\insertsubtitle}
  }
}

\section{Introduction}
\frame{
\textcolor{ecs100}{\huge{Introduction}}\\[0.2cm]
}

\frame{
\frametitle{Introduction}
\begin{itemize}
   \item <2-> Primary aim: to reimplement the basic FEM package using the Flexible Library for Efficient Numerical Solutions (FLENS).
   \begin{itemize}
   	\item[$\hookrightarrow$] <2-> FLENS offers implementations of matrix/vector storage types, together with the associated linear algebra operations.
   	%\item[$\hookrightarrow$] <3-> As well as standardisation, FLENS offers several
   \end{itemize}
   \item[] <2->
   \item <3-> Objectives:
   \begin{enumerate}
   	\item <4-> Replace all data storage objects in the FEM package with FLENS-based objects.
   	\vspace{0.3cm}
   	\item <5-> Replace linear algebra operations with BLAS equivalents.
   	\vspace{0.3cm}
   	\item <6-> Offer two versions of solvers, one using BLAS notation, one using overloaded operators.
   \end{enumerate}
\end{itemize}
}

\section{Introduction}
\frame{
\textcolor{ecs100}{\huge{Replacing data storage objects}}\\[0.2cm]
}

\frame{
\frametitle{Replacing data storage objects}
\begin{itemize}
	\item <2-> Some objects can be directly converted:
	\begin{equation*}
	\begin{array}{rcl}
   	\mbox{\texttt{Vector}}  &  \rightarrow  &  \mbox{\texttt{DenseVector\<Array\<\double\> \>}} \\
   	\mbox{\texttt{IndexVector}} & \rightarrow & \mbox{\texttt{DenseVector\<Array\<\int\> \>}}
	\end{array}
	\end{equation*}
	\item[] <2->
	\item <3-> FLENS offers a CRS matrix implementation, but it must be initialised from a sparse matrix using the coordinate storage scheme.
\end{itemize}
}

\begin{frame}[fragile]
\frametitle{The FLENSDataVector}
\begin{itemize}
	\item <2-> The biggest change was the \texttt{DataVector} storage class, since it contains additional objects and methods for the parallel MPI communications.
	\vspace{0.2cm}
	\item <3-> We replaced this class with our own \texttt{FLENSDataVector} class:
	\begin{itemize}
		\item[$\hookrightarrow$] <3-> Derived from a FLENS \texttt{DenseVector}:
	\end{itemize}

\item[] <3->\begin{lstlisting}
template <typename VTYPE = FLvNonMPI>
struct FLENSDataVector
        : public DenseVector<Array<double> 
{ ... }
\end{lstlisting}

	\item[] <4-> 
	\begin{itemize}
		\item[$\hookrightarrow$] Contains almost all objects and methods from the \texttt{DataVector}:
	\end{itemize}

\begin{lstlisting}
	const Coupling &coupling;

	void typeII_2_I();
	void typeI_2_II();
	
	void commCrossPoints();
	void commBoundaryNodes();
\end{lstlisting}
\end{itemize}
\end{frame}
		
\begin{frame}[fragile]
\frametitle{The FLENSDataVector}
\begin{itemize}
	\item <2-> Instead of the MPI vector type as a member enumerated object, the type is specified (permanently) at instantiation:
	\begin{lstlisting}
//Vector types:
class FLvNonMPI;
class FLvTypeI;
class FLvTypeII;
	\end{lstlisting}
	\item[] <3-> 
	\begin{lstlisting}
FLENSDataVector<FLvTypeI> myVector(20, myCoupling);
\end{lstlisting}
	\item <4-> Clever specialisations of constructors prevents inappropriate usage.
	\item[] <5->
	\vspace{0.3cm}
	\begin{itemize}
		\item[$\hookrightarrow$] More rigorous.
		\vspace{0.1cm}
		\item[$\hookrightarrow$] <6-> Makes many assertions redundant, as all type checking is done at compile time.
	\end{itemize}
	
\end{itemize}
\end{frame}
	
\section{Introduction}
\frame{
\textcolor{ecs100}{\huge{BLAS and the Conjugate Gradient solver}}\\[0.2cm]
}

\begin{frame}[fragile]
\frametitle{BLAS}
\begin{itemize}
	\item <2-> Basic Linear Algebra Suite
	\vspace{0.3cm}
	\begin{itemize}
		\item[$\hookrightarrow$] Included in FLENS.
		\vspace{0.3cm}
		\item[$\hookrightarrow$] Available via overloaded operators in FLENS.
	\end{itemize}
	\vspace{0.3cm}
	\item <3-> GotoBLAS / OpenBLAS / ATLAS
	\vspace{0.3cm}
	\begin{itemize}
		\item[$\hookrightarrow$] Basically `supercharged' BLAS routines, usually optimised for a specific processor.
		\vspace{0.3cm}
		\item[$\hookrightarrow$] Can be linked into FLENS (to replace FLENS BLAS routines) trivially\\ (one compiler flag + linker paths).
	\end{itemize}
\end{itemize}
\end{frame}

\begin{frame}[fragile]
\frametitle{BLAS}
\begin{itemize}
	\item <1-> FEM Package function call:
	\begin{lstlisting}
CRSmatVec(p,A,x);
add(r2,p,-1.);
	\end{lstlisting}
\end{itemize}
\end{frame}

\begin{frame}[fragile]
\frametitle{BLAS}
\begin{itemize}
	\item <1-> Our function call via BLAS:
	\begin{lstlisting}
blas::mv(NoTrans, 1., A, x, 0., p); 
blas::axpy(-1., p, r2);
	\end{lstlisting}
\end{itemize}
\end{frame}

\begin{frame}[fragile]
\frametitle{BLAS}
\begin{itemize}
	\item <1-> Our function call via overloaded operators:
	\begin{lstlisting}
p = A*x;
r2 = r2 - p;
	\end{lstlisting}
\end{itemize}
\end{frame}

\begin{frame}[fragile]
\frametitle{BLAS}
\begin{itemize}
	\item <-1> Care was taken when overloading BLAS operations that require MPI communications:
	\begin{lstlisting}
//Overloaded dot - performs appropriate communication:
double
dot(const FLENSDataVector<FLvTypeI> &x1, 
    const FLENSDataVector<FLvTypeII> &x2)
{
  //Upcast to DenseVector, and use the standard blas::dot:
  double value = 
      blas::dot(
        *static_cast<const DenseVector<Array<double> > *>(&x1),
        *static_cast<const DenseVector<Array<double> > *>(&x2)
      );

  double v = 0;
  //*** Communication to add values from other processes ***/
  MPI::COMM_WORLD.Allreduce(&value,&v,1,MPI::DOUBLE,MPI::SUM);

  return v;
}
	\end{lstlisting}
\end{itemize}
\end{frame}

\frame{
\begin{figure}
      \centering
      \newlength\figureheight
      \newlength\figurewidth
      \setlength\figureheight{6cm}
      \setlength\figurewidth{10cm}
      % This file was created by matlab2tikz v0.4.0.
% Copyright (c) 2008--2013, Nico Schlömer <nico.schloemer@gmail.com>
% All rights reserved.
% 
% The latest updates can be retrieved from
%   http://www.mathworks.com/matlabcentral/fileexchange/22022-matlab2tikz
% where you can also make suggestions and rate matlab2tikz.
% 
% 
% 

% defining custom colors
\definecolor{mycolor1}{rgb}{1,0,1}%

\begin{tikzpicture}

\begin{axis}[%
width=\figurewidth,
height=\figureheight,
scale only axis,
xmode=log,
xmin=10000,
xmax=10000000,
xtick={10000,100000,1000000,10000000},
xminorticks=true,
minor x tick num={3},
xlabel={Mesh Element Count},
ymode=log,
ymin=0.1,
ymax=100,
ytick={0.1,1,10,100},
yminorticks=true,
minor y tick num={3},
ylabel={Computation time},
title={Serial vs. Parallel for System Assembly},
axis on top,
legend style={at={(0.03,0.97)},anchor=north west,draw=black,fill=white,legend cell align=left}
]
\addplot [
color=red,
solid,
line width=2.6pt,
mark size=3.0pt,
mark=x,
mark options={solid}
]
table[row sep=crcr]{
33025 0.817531\\
131585 3.41866\\
525313 14.7845\\
2099201 65.8931\\
};
\addlegendentry{Serial};

\addplot [
color=green,
solid,
line width=1.3pt,
mark size=3.0pt,
mark=x,
mark options={solid}
]
table[row sep=crcr]{
33025 0.827823\\
131585 3.54683\\
525313 15.4809\\
2099201 67.2637\\
};
\addlegendentry{Serial with GotoBLAS};

\addplot [
color=blue,
solid,
line width=2.6pt,
mark size=3.0pt,
mark=x,
mark options={solid}
]
table[row sep=crcr]{
33284 0.195\\
132100 0.824\\
526340 3.57\\
2101252 15.27\\
8396804 65\\
};
\addlegendentry{Parallel};

\addplot [
color=mycolor1,
solid,
line width=1.3pt,
mark size=3.0pt,
mark=x,
mark options={solid}
]
table[row sep=crcr]{
33284 0.204\\
132100 0.845\\
526340 3.68\\
2101252 16.01\\
8396804 68.1\\
};
\addlegendentry{Parallel with GotoBLAS};

\end{axis}
\end{tikzpicture}%
      \label{fig:assembly}
\end{figure}
}

\frame{
\begin{figure}
      \centering
      \setlength\figureheight{6cm}
      \setlength\figurewidth{10cm}
      % This file was created by matlab2tikz v0.4.0.
% Copyright (c) 2008--2013, Nico Schlömer <nico.schloemer@gmail.com>
% All rights reserved.
% 
% The latest updates can be retrieved from
%   http://www.mathworks.com/matlabcentral/fileexchange/22022-matlab2tikz
% where you can also make suggestions and rate matlab2tikz.
% 
% 
% 

% defining custom colors
\definecolor{mycolor1}{rgb}{1,0,1}%

\begin{tikzpicture}

\begin{axis}[%
width=\figurewidth,
height=\figureheight,
scale only axis,
xmode=log,
xmin=10000,
xmax=10000000,
xtick={10000,100000,1000000,10000000},
xminorticks=true,
minor x tick num={3},
xlabel={Mesh Element Count},
ymode=log,
ymin=0.1,
ymax=10000,
ytick={0.1,1,10,100,1000,10000},
yminorticks=true,
minor y tick num={3},
ylabel={Computation time},
title={Serial vs. Parallel for System Solving},
axis on top,
legend style={at={(0.03,0.97)},anchor=north west,draw=black,fill=white,legend cell align=left}
]
\addplot [
color=red,
solid,
line width=2.6pt,
mark size=3.0pt,
mark=x,
mark options={solid}
]
table[row sep=crcr]{
33025 1.49912\\
131585 12.4984\\
525313 101.033\\
2099201 802.526\\
};
\addlegendentry{Serial CG};

\addplot [
color=green,
solid,
line width=1.3pt,
mark size=3.0pt,
mark=x,
mark options={solid}
]
table[row sep=crcr]{
33025 1.02973\\
131585 9.27832\\
525313 78.1647\\
2099201 620.832\\
};
\addlegendentry{Serial CG with GotoBLAS};

\addplot [
color=blue,
solid,
line width=2.6pt,
mark size=3.0pt,
mark=x,
mark options={solid}
]
table[row sep=crcr]{
33284 0.8422\\
132100 6.5692\\
526340 54.327\\
2101252 459.899\\
8396804 3641.48\\
};
\addlegendentry{Parallel CG};

\addplot [
color=mycolor1,
solid,
line width=1.3pt,
mark size=3.0pt,
mark=x,
mark options={solid}
]
table[row sep=crcr]{
33284 0.551\\
132100 4.468\\
526340 40.657\\
2101252 329.28\\
8396804 2589.78\\
};
\addlegendentry{Parallel G with GotoBLAS};

\end{axis}
\end{tikzpicture}%
      \label{fig:solver}
\end{figure}
}

\section{The End}
\frame{
\textcolor{ecs100}{\huge{The End}}\\[0.2cm]
}



\end{document}